\documentclass{report}

%-----------------------------------
%--- Paul default latex header---
%-----------------------------------

%---PACKAGES---

\usepackage[Glenn]{fncychap}

\usepackage{fancyhdr}

\usepackage[utf8x]{inputenc} 
\usepackage[T1]{fontenc}      
\usepackage[french]{babel} 

\usepackage{array}

\usepackage{mathtools}
\usepackage{amssymb}
\usepackage{mathrsfs}
\usepackage{mathabx}

\usepackage{xcolor}
\usepackage{graphicx}

\usepackage[a4paper]{geometry}
\geometry{hscale=0.85,vscale=0.85,centering}

%---HEAD---

\title{Projet Prog 1\\ou\\L'envie de vaincre}
\author{Paul Bigot}
\date{30 octobre 2022}

\renewcommand\thesection{\arabic{section}}

\pagestyle{fancy}
\fancyhf{}
\lhead{Prog1}
\chead{L'envie de vaincre}
\rhead{Paul Bigot}
\cfoot{\thepage}

\begin{document}

\maketitle
\section{Préambule}
Avant toute chose, je tenais à prévenir le lecteur qu'il n'est pas sur le point de lire un simple rapport de Projet Prog, mais bien un passage marquant de ma vie. En effet, sont relatées dans cet ouvrage quelques expériences marquantes de ma vie, quelques difficultées qu'il m'a fallu surmonter pour parvenir là où je suis aujourd'hui, et celles-ci peuvent être traumatisantes à écouter, c'est pourquoi il me semble nécessaire de préciser que malgré ces passages difficiles, j'ai retrouvé l'intégralité de mon intégrité mentale, mon envie de vaincre est actuellement à son maximum et ce sont ces passages difficiles qui m'ont forgés. Ce sont ces expériences douloureuses qui vous font grandir malgré la souffrance qu'elles vous infligent, qui vous donnent une rage de bataille incroyable, aussi bas puissent-elles vous emporter, qui vous prodiguent des leçons de vies très profondes malgré les ténèbres intelectuelles qu'elles vous font connaître, qui ont fait de moi la personne que je suis actuellement, ainsi, je ne regrette rien et je dois même dire que je suis fier du chemin parcouru bien que cela ne soit qu'encore une faible portion de la route menant à l'accomplissement personnel le plus total et c'est une partie de ce chemin que vous êtes sur le point de découvrir...


\section{Le choc de l'épreuve}
Lorsque l'on nous a annoncé que nous avions un Projet Prog à réaliser, je fus immédiatemment pris de cours et l'incompréhension me submergea tout d'abord: j'allais devoir réaliser un compilateur en Assembleur, un language que je ne comprenais pas et qui me paraissait totalement incompréhensible. Avais-je seulement la moindre chance d'y arriver, moi? J'en doutais fortement, heureusement, comme ma mère me le disais souvent: ``Si tu ne peux escalader une montagne, contourne-là!'', j'ai donc suivi son conseil et j'ai cherché comment contourné, puis j'ai trouvé: il me suffisait de ne pas rendre ce projet et de me concentrer sur les autres matières, si j'avais 0/20 en Prog, il me suffisait d'avoir 15/20 en algo et en calculabilité pour tout de même valider le Bloc 1 et ainsi valider le diplôme. Mais voilà le problème: alors que j'étais le point de fêter ma victoire grâce à mes calculs sournois, j'appris qu'il était nécessaire d'obtenir la note de 10/20 aux deux projets de Programmation pour valider. Ce fut un choc terrible, j'eus alors la sensation d'être poignardé dans le dos et cela me rappela cette soirée d'automne où, me promenant avec mon père, nous nous fîmmes attaqué par des brigands, mon père tenta de se battre tant qu'il put, il en mis un K.O. à l'aide d'une droite bien placée, en mis un deuxième hors d'état de nuire à l'aide d'un coup de pied, il pensa alors qu'il prenaint le dessu sur la sitution, ce qui lui fit baisser sa garde une fraction de seconde, ce qui fut suffisant pour que le troisième le poignarde tandis que je prenais la fuite. Depuis ce jour, je m'étais promis de ne jamais considérer commme gagné ce que je n'avais pas encore remporté, car comme le dis Sun Tzu dans l'Art de la Guerre : ``Il n'y a de plus faible combattant que celui qui n'est plus en guerre.'' Ce sont ces paroles qui me permirent de rester debout à ce moment là. Cependant, je dus me rendre à l'évidence: j'étais incapable de réaliser ce projet. Il me fallait donc trouver une solution, j'allais être viré de l'ENS, qu'allais-je faire ensuite? Il me sembla alors que plusieurs solutions étaient envisageables: j'avais entre autres envisagé de rejoindre le GIGN, ou encore de repasser les concours pour aller dans une autre école. Mais ce n'était qu'un retardement illusoir, si j'échouais maintenant à l'ENS, pourquoi est-ce je pourrais réussir ailleurs plus tard? Il fallait se rendre à la raison, les personnes qui se moquaient de moi à l'école primaire Victor Hugo parceque j'avais perdu mon père avaient toujours eu raison: je n'étais qu'un raté sans avenir, un bon à rien. Je commença à repenser en boucle leurs moqueries dans ma tête et à perdre toute motivation, jusqu'à finir dans une dépression, qui dura quelques mois, mon état mental s'aggravant jour après jour jusqu'au point qu'il me paraissait quasiment impossible de ressortir de cette situation...

\section{Le regain d'espoir}
Alors que j'étais sur le point de sombrer définitivement dans la dépression, je me souvint soudainement des paroles profondes du grand Youtuber français Tibo InShape : ``Rien à foutre de ta dépression!!!''. Alors que ces paroles auraient pu m'enfoncer encore plus, elles me firent au contraire comprendre l'origine de mes problèmes: ce n'était pas mes capacités mais plutôt ma mentalité. Ainsi ma dépression avait disparu et je me souvint des phrases que mon père me répétait avant sa mort : ``Bat toi pour tes rêves avant que tes rêves ne te battent'' ou encore ``Peu importe la tournure des événements, si tu doutes de toi alors tu as perdu avant même que la bataille n'ait commencée''. Je me remis alors en question, je m'étais toujours dit que jamais je n'arriverais à coder en Assembleur, mais c'était faux, il suffisait de développer mes capacités. Si je ne comprends pas, il me suffirait d'apprendre, si on me proliférait des insultes,  je les ignorerais, si on me mettais une barrière, je la briserais et si on me mettais un mur, je l'escaladerais. Je me sentais requinqué et après avoir médité pendant 3 jours et 3 nuits sans interruption, j'étais prêt à attaquer ce projet prog, tel un prédateur se préparant à se ruer sur sa proie. 
Je commençais donc par le Lexer, je définis alors un premier type de lexème qui permettait d'interpréter une chaine de caractère, caractère par caractère, en enlevant les espaces, ce qui avait comme problème de considérer comme équivalentes les chaînes ``\verb|  9  2.      3*     .45     .2|'' et ``\verb|92.3 *. 45.2|'', heureusement personne n'écris la première chaîne naturellement, et je ne pris donc pas la peine de régler ce problème, le jugeant trop futile. Cependant le type de lexème que j'avais défini était trop peu performant, je dus donc me remettre en question, c'était un nouveau coup porté à mon ego, mais cela ne m'importait plus, tout ce que je désirais c'était m'améliorer. Je voulais devenir le meilleur créateur de Lexer, je redéfinis donc un nouveau type de lexème, plus polyvalent que le précédent, et après quelques heures de travail acharné j'arrivais finalement à transformer une chaîne de caractère en une liste de lexèmes améliorés, j'étais, à l'aide de mes capacités intelectuelles venu à bout du lexer!!!
\section{De nouveau des doutes?}
Une fois le lexer terminé, il me fallut donc passer au parser, et
les choses se retrouvèrent un peu plus compliquées qu'elles ne l'étaient auparavant. En effet mon objectif final était de parvenir à un arbre syntaxique, 
et tout d'abord le nom posait déjà problème: un arbre ça oui, 
je savait ce que c'était, la syntaxe aussi mais un 
arbre syntaxique était un terme bien plus obscur à mes yeux. 
Après réflexion je compris ce que l'on attendait de moi et de la forme de l'arbre à obtenir. 
Je me lançais donc dans la quête de la création de cet arbre en créant un arsenal de nouvelles fonctions me permettant d'obtenir le type d'arbre que je désirais. Cependant, cette mission fut perturbée par des voix que je percevais, sans que je ne puisse distinguer si elles étaient prononcées par certains de mes camarades ou par des diablotins maléfiques, probablement un peu des deux: \verb|``utilise ocamllex''|, \verb|``utilise ocamlyacc''|, \verb|``librairie x86-64''|, mais je résistais à ces tentations en me souvenant de ce que ma mère me disais souvent: ``Nul n'est plus puissant que celui qui peut tout refuser'', mais je voyais bien que certains de mes camarades avaient sombré du côté obscur: ils paraissaient plus mécaniques, moins humains et plus diaboliques qu'auparavant.
Quant à moi, je restais fidèle à mes convictions et bien qu'il m'arriva de rencontrer des difficultés, j'appris de celles-ci pour devenir encore plus puissant, et j'en vint à définir mon type arbre selon trois cas possibles:
\begin{itemize}
 \item \verb|Une Feuille| : c'est le cas lorsque l'on tombe sur un entier ou un flottant, dans ce cas nous sommes sur une extrémité de l'arbre
 \item \verb|Un Node| : c''est le cas lorsque l'on tombe sur un opérateur nécessitant deux arguments, l'opérateur prend alors la place du node et chacune des deux expressions en argument devient un arbre fils du node
 \item \verb|Une Branche| : c'est comme un node mais avec un unique fils, et donc avec un opérateur unaire, c'est-à-dire float, int ou - devant une expression
\end{itemize}
Cette partie fut, je dois l'admettre, plutôt laborieuse, d'autant plus que bon nombre de mes camarades insensibles à mon courage et à mon honnêteté se moquaient de moi, surement touchés par l'épidémie de diabolicité qui faisait rage suite à l'utilisation abusive d'ocamllex et d'ocamlyacc, mais cela ne m'a pas découragé pour autant, il m'a suffi d'ignorer leurs piètres balbutiements et de continuer sur ma lancée.
De plus, après avoir passé plusieurs heures pour produire mon code, je me rendis compte qu'il ne marchait pas dans ceratins cas spécifiques, par exemple lorsque l'on avait quelquechose de la forme \verb|-(exp)| ou \verb|+(+(...)...)|, mais je ne perdis pas espoir, et lors de mes quelques moments de doutes, je repensais à mon père qui me disait ``Les seuls soldats vaincus ce sont ceux qui sont morts, et ceux qui se sont rendus'', et je ne lachais rien car je voulais être un battant, quelqu'un dont mon père aurait été fier, et je développa cette capacité particulière que je nommerais ``L'envie de vaincre''. A force de persévérance et de courage, je finis par réussir à complètement déboguer mon code qui me renvoyait donc des arbres magnifiques. Je pouvais être fier de moi, j'avais réussi à planter, à faire pousser, à éduquer, et à faire se développer ces arbres par moi-même, c'était ma création, tandis que nombreux étaient ceux qui, parmi mes camarades, s'étaient contentés d'aller rammassés le premier arbre qu'ils avaient trouvés dans la forêt, finissant tous ainsi avec des arbres se ressemblant.

\section{Une aide précieuse}
Me voilà enfin à la partie que je redoutais tant, transformer mon arbre en code Assembleur. Je dois avouer que devoir transformer ces arbres si Magnifiques en code Assembleur si horrible m'a au départ brisé le cœur, mais, comme dit le proverbe: ``Il est bien bête de mourir de faim après avoir passé 20 ans de sa vie à faire pousser un verger''. N'ayant pas fait tout ce chemin pour rien, et étant plus motivé que jamais pour valider, j'ai donc du sacrifier mes arbres, c'est ainsi, la vie est parfois cruelle mais il est parfois nécessaire de faire des sacrifices pour avancer. Cependant il restait un problème majeur, je ne savais toujours pas comment diable on pouvait coder en Assembleur. Je me mis donc à essayer d'apprendre par moi-même toutes les subtilités du code Assembleur, et, au bout d'une semaine de travail à un rythme de 15 heures par jour, je peinais encore à faire une addition d'entier. Bien que la situation était préoccupante, je ne laissais pas mes émotions me submerger, ainsi, ayant appris de mes erreurs, je ne laissais alors pas la panique et le doute m'emparer, et je tentais de garder mon sang froid afin de rester le plus lucide possible dans ces moments pourtant inquiétants, et c'est alors que je reçu l'aide de mes brillants camarades qui m'expliquèrent le code Assembleur et bien plus encore. Bien qu'ils avaient leurs défauts respectifs, je me souviendraient toujours de la manière dont ils ont su se rendre disponible pour moi, m'éclairant de leur génie. J'étais alors tel un aveugle voyant pour la première fois, éclairé par une lumière divine. Rémi Baron, Hugo Fruchet, ou encore Simon Corbard pour ne citer que ces trois là bien qu'il y en ai eu d'autres, surent m'expliquer mieux que quiconque le fonctionnement de la pile en Assembleur, les subtiles différences entre les entiers et les flottants, les casse-têtes des fonctions d'affichage, l'intérêt de créer un makefile, ou encore m'expliquer commment compiler mon code, des problèmes qui peuvent paraître triviaux pour certains, mais qui étaient pour moi des montagnes infranchissables. Je ne saurais comment remercier ces vaillants combattants pour l'aide qu'ils m'ont apporté et j'espères que les chemins de nos destinés respectives se croiseront à nouveau dans le futur afin que je puisse leur rende la pareille.

\section{La fin de cette expérience si particulière et les enseignements que j'ai pu en tirer}
Grâce à cette aide si précieuse, je gagna immédiatemment en puissance et j'attaqua la fin du Projet Prog de manière enthousiaste, au point que ce qui me parraissait quelque jours plus tôt être un océan peuplé de requins n'était plus pour moi qu'une simple flaque d'eau, ce qui me permis de réaliser que la difficulté des tâches à accomplir n'est qu'une invention de notre cerveau, présente pour nous effrayer et nous ralentir, mais il ne faut pas freiner, il faut laisser ses capacités s'exprimer pleinement, et c'est ainsi que ma confiance en moi a pu atteindre son paroxysme à la fin du projet prog, et que malgré les quelques difficultés restantes, à savoir, quelques bugs résiduels, mon ignorance de LateX et mon incompréhension du makefile, j'ai me suis à nouveau surpassé, en pensant à mon père, à ma mère, et à toutes les autres personnes qui ont toujourscru en moi et j'ai pu finalemnt triomphé de ce projet Programmation. Mais au final, il me semble que le plus impportant ce n'est pas la réussite de ce projet bien que ça me permette de valider, c'est avant tout les leçons que l'on en tire, les leçons des erreurs commises, ce qui nous permet de progresser afin de nous sublimer toujours plus et de laiser s'exprimer au mieux notre potentiel qui n'est d'ailleurs probablement borné que par les limites que nous nous fixons nous-mêmes, mais c'est aussi l'importance des personnes qui nous ont aidés dans notre quête.
Finalement, les enseignement que je relevrais de mon histoire et que je shouterais donner à tout le monde sont les suivants:
\begin{itemize}
 \item ``Seules tes croyances limitantes, et les pensées ralentissantes de ton cerveau t'empêchent de réaliser tes rêves''
 \item ``Bien que la route vers ta destinée t'es totalement personnelle, tu avancera plus vite en avançant aux côtés des autres''
 \item ``Développe en toi cette envie de vaincre, car, uns fois qu'ellle se sera étainte, la vie t'aura vaincu''
\end{itemize}
Bien évidemment, libre à chacun de tirer ses propres enseignements de mon histoire ainsi que de la sienne ou de celle de quiconque, mais si moi, Paul Bigot, ai réussi à venir à bout de ce projet prog, cela montre bien que rien n'est impossible!

\end{document}
